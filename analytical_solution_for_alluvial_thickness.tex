\documentclass[11pt]{amsart}
\usepackage{geometry}                % See geometry.pdf to learn the layout options. There are lots.
\geometry{letterpaper}                   % ... or a4paper or a5paper or ... 
%\geometry{landscape}                % Activate for for rotated page geometry
%\usepackage[parfill]{parskip}    % Activate to begin paragraphs with an empty line rather than an indent
\usepackage{graphicx}
\usepackage{amssymb}
\usepackage{epstopdf}
\DeclareGraphicsRule{.tif}{png}{.png}{`convert #1 `dirname #1`/`basename #1 .tif`.png}

\title{Analytical solution for $H(t)$ in SPACE model}
\author{GT}
\date{June 2016}                                           % Activate to display a given date or no date

\begin{document}
\maketitle

The evolution equation for alluvial thickness at a node is:
\begin{equation}
\frac{d\hat{H}}{dt'} = \hat{D} - \hat{E}_s
\end{equation}
where $\hat{D}$ is deposition rate and $\hat{E}_s$ is entrainment rate. The latter is $\hat{E}_s = K_s q S (1-\exp(-\hat{H}/H_*)$. We want a solution for $\hat{H}(t)$ for a small interval of time during which $\hat{D}$ and $K_s q S$ are constant in time.

Start by nondimensionalizing as follows:
\begin{eqnarray}
H = \hat{H} / H_* \\
D = \hat{D} / K_s q S \\
t = t' K_s q S / H_*
\end{eqnarray}

When we plug these definitions in the evolution equation, we get
\begin{equation}
\frac{dH}{dt} = D - 1 + e^{-H}
\end{equation}
Separate variables and do an indefinite integral:
\begin{equation}
\int \frac{1}{D - 1 + e^{-H}} dH = \int dt
\end{equation}
This integral evaluates to (according to online integrator):
\begin{equation}
\frac{1}{D-1} \ln \left[ e^{H} (D-1) + 1 \right] = t + C
\end{equation}
To evaluate $C$, note that when $t=0$ (the beginning of a time step), $H=H_0$, where $H_0$ just means the initial value at the start of the time step. Multiply both sides by $(1-D)$:
\begin{equation}
\ln \left[ e^{H} (D-1) + 1 \right] = (D-1)t + (D-1) C
\end{equation}
Apply the exponential function to both sides:
\begin{equation}
e^H (D-1) + 1= \exp( (D-1)t + (D-1) C ) = e^{(D-1)t} e^{(D-1)C}
\end{equation}
Substitute in for $C$ and let the $e$ undo the $\ln$:
\begin{equation}
e^H (D-1) + 1= e^{(D-1)t} \left( (D-1) e^{H_0}  + 1 \right)
\end{equation}
Subtract one from both sides, and divide both by $(D-1)$:
\begin{equation}
e^H = \frac{1}{D-1} \left( e^{(D-1)t} \left( (D-1) e^{H_0}  + 1 \right) - 1 \right)
\end{equation}
Apply the natural log function to both sides:
\begin{equation}
H(t) = \ln \left[ \frac{1}{D-1} \left( e^{(D-1)t} \left( (D-1) e^{H_0}  + 1 \right) - 1 \right) \right]
\end{equation}
YAY!

But note this blows up for $D=1$. That requires a special, and simpler solution. If $D=1$, then
\begin{equation}
\frac{dH}{dt} = e^{-H}
\end{equation}
Which, when applying the rule $H=H_0$ at $t=0$, integrates to:
\begin{equation}
H(t) = \ln \left[ t + e^{H_0} \right]
\end{equation}

We can also re-dimensionalize these. For $D=1$, or in other words $\hat{D} = K_sqS$, 
\begin{equation}
\hat{H}(t') = H_* \ln \left[ \frac{K_sqS}{H_*} t' + e^{\hat{H}_0/H_*} \right]
\end{equation}
Otherwise,
\begin{equation}
\hat{H}(t) = H_* \ln \left[ \frac{1}{D-1} \left( e^{(\hat{D}-K_s q S)t' / H_*} \left( (D-1) e^{\hat{H}_0/H_*}  + 1 \right) - 1 \right) \right]
\end{equation}
where I've kept the abbreviated $D$ for simplicity in a couple of places.

There's a time scale: $\tau = H_* / (\hat{D}-K_sqS)$.

%\section{}
%\subsection{}



\end{document}  